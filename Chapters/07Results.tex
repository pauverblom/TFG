\chapter{Results}
\label{chap:Results}

In this chapter, you describe the output from applying your methodology. Present your results in a clear manner, using a combination of figures and tables.

If you are doing lab experiments: What do you observe? Also include qualitative observations that might not be directly relevant (“we saw more production at the upper half of the core sample than the lower half, with most production leaving at the upper side-plane”).

If you are doing computations: What is the output from applying your code/software? Which part of your software is using the most computational power? How does it compare to other similar software?

Illustrate the results. Use graphs, images, and tables (see Chapter \ref{chap:Theory}).

A common challenge is to distinguish the results and the discussion section. Do not discuss your results in this chapter, that should be done in the discussion section. If it is hard to separate results and discussion, you might combine them into one section. You could also split the results and discussion part of your thesis into several results/discussion sections for different topics (“Magnesium effect on imbibition”, “Sodium effect on imbibition”).

Try to avoid repetition between. In case you have many graphs of the same process, keep some characteristic graphs and move the rest to an appendix. Then you describe one graph in detail and refer to the others in the appendix for changes between the results.