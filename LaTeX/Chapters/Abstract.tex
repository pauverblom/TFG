Active Galactic Nuclei (AGN) are some of the brightest objects in the universe, often outshining their entire host galaxy \citep{RadiativeProcesses}. Comprehending the intricate structure and workings of these objects has driven decades of astronomical study. More recently, there has been growing interest in the role of high-energy neutrinos in these objects, as exemplified by the work of \citet{Eichmann_2022}. The IceCube collaboration \citep{IceCube2022} has detected a significant neutrino flux from the direction of NGC 1068, a Seyfert II galaxy located at a distance of approximately 10.1 Mpc. This has led to the conclusion that NGC 1068 is a source of high-energy neutrinos, produced alongside comparably energetic gamma rays. However, the observed gamma-ray luminosity in the GeV - TeV range is significantly lower than expected, as reported by the Fermi Gamma-ray Space Telescope in its 12-year survey \citep{Fermi12yeardata}, prompting investigation into the cause of this discrepancy.

In this work, a model describing the propagation and reprocessing of high-energy gamma rays within the AGN, based on electromagnetic cascading, has been developed. For best-fit parameters, the model predicts a low-energy gamma-ray excess from reprocessed cascade photons in the $10^6-10^8$ eV range, which could be probed by future observations. The model also constrains the coronal radius to lie within $1-50$ Schwarzschild radii, consistent with theoretical expectations.