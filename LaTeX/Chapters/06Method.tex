\chapter{Methods}

In this chapter, you describe the methods used to obtain your results. This can be new methods for analyzing existing data, or existing methods applied to new data. Include a complete description of your methods. It is often helpful with flow charts to explain your methods.

Give enough details for others to be able to reproduce your work. Do not overdo it, as a rule of thumb you can assume that the reader has the same background as you had when you started the thesis.

Introduce and describe all parameters that have been tested in your project, and why these in particular have been varied. What was the reason for varying these parameters?

Additional details can be given in the appendices. Appendices are useful to avoid too much information in the thesis itself, which can be detrimental to the reading experience.

If you are developing or using software, then it is common to include pseudo-codes for the software. The full code can be added as an appendix, but it is even better to upload the code to a public repository (e.g., \url{github.com}), and link the repository from the appendix.

