\chapter{Introduction}
\label{chap:Introduction}

\section{Motivation}

In recent years, there has been a growing interest in the role that neutrinos play in high-energy astrophysical phenomena. Among these, Active Galactic Nuclei (AGN) have been identified as significant sources of high-energy particles, including gamma rays and neutrinos. One particular AGN, NGC 1068, has garnered attention due to its unique gamma-ray emission profile. Despite the expectation that the gamma-ray luminosity should be comparable to the energy of the neutrinos produced in the same astrophysical processes, observations have shown that the gamma-ray emission from NGC 1068 is significantly lower in energy.

This discrepancy raises several intriguing questions about the underlying mechanisms governing the emission processes in AGN. Understanding why the gamma-ray luminosity received from NGC 1068 is of such low energy compared to the neutrinos is crucial for advancing our knowledge of particle acceleration and interaction in these extreme environments. It suggests that there may be additional factors or processes at play that are not yet fully understood.

The motivation for this project stems from the need to model the gamma-ray emission of NGC 1068 accurately. By developing a comprehensive model, we aim to identify the factors contributing to the observed low-energy gamma-ray emission. This includes investigating potential absorption mechanisms, energy dissipation processes, and the role of magnetic fields and other environmental conditions within the AGN. Through this research, we hope to provide a clearer picture of the high-energy astrophysical processes in AGN and contribute to the broader understanding of cosmic particle acceleration and emission mechanisms.

\subsection{Research question}

The research question you are trying to answer through your thesis should be formulated in the introduction section. The purpose of the motivation should lead to a hypothesis, and the research questions should be formulated so that it can verify the hypothesis.

You can also formulate a set of objectives for your work to answer your research question.  It should be clear how these objectives together will answer your research question.

\section{Outline}

It is common to end the introduction with an outline of the thesis. Here you could briefly present the different chapters and their content.