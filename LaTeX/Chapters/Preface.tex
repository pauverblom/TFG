The universe presents us with incredible extremes: temperatures, energies, and distances that defy everyday intuition. Perhaps most captivating among these are Active Galactic Nuclei, galactic cores housing supermassive black holes and emitting luminosities that can outshine their entire host galaxies. Initially, my interest in general astrophysics motivated me to seek a project in this research area. However, upon learning of the opportunity to work on these extreme objects, I became immediately interested. As I delved deeper into AGNs, I was struck by the complexity of their environments and the variety of phenomena they exhibit. The more I learned, the more I wanted to understand.

This project began with a simple but intriguing question: why would a galaxy like NGC 1068 be capable of producing high-energy neutrinos if not accompanied by the gamma-ray signal one would expect from the same pion decay? As I explored further into this anomaly, I discovered rich interaction between theory and observation, between what we see and what must be hidden. This led to the development of a model incorporating electromagnetic cascades to explain the attenuation of gamma rays while allowing high-energy neutrinos to escape.

As I wrote this work, I sensed the inspiration of the multi-messenger astronomy ethos—the idea that the universe is singing in many a voice, and that in listening to more than one, we better understand cosmic events. I hope that this research contributes modestly to the growing understanding of the high-energy cosmos and the powerful engines at galactic centers.

I am earnestly grateful to Dr. Kachelrie\ss, whose counsel and advice were invaluable for this work. Likewise, I also extend my gratitude to my family and friends, for their moral support and encouragement during the development of this project.